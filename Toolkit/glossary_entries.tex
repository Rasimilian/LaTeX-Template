% ----- Combined Entries ----- %
% Argument order: id, acronym, acronym written out, description
% Create both acronym and glossary entry with linking:
\newcommand{\newglossaryacronym}[4]{
    \newacronym{#1acr}{#2}{\acrlong{#1}}
    \newglossaryentry{#1}{
        name={#2 (#3)},
        text={#2},
        % symbol={#2},
        short={#2},
        long={#3},
        first={#3 (#2)},
        firstplural={#3\glspluralsuffix~(#2\glspluralsuffix)},
        description={#4\glsadd{#1acr}}
    }
}

% ----- Nomenclature ----- %

\newglossaryentry{detail}{
    name={detail},
    text={detail},
    description={Additional info}}
}

\newglossaryentry{atomic-mass}{
    name={atomic-mass},
    text={atomic-mass},
    description={Mass of an atom}}
}

% ----- Detailed Acronyms, they show up in the nomenclature ----- %

\newglossaryacronym{SRT}{SRT}{Special Relativity Theory}{
    Special Relativity Theory
}

% ----- Acronyms ----- %

\newacronym{}{}{}

% ----- Symbols ----- %

\newglossaryentry{M}{
    type=\symboltype,
    sort={M},
    name={\(\mathbf{ M }\)},
    text={Mass},
    symbol={\(M\)},
    description={Mass. Unit: \unit{\kilogram}}
}

\glsenableentrycount                             % Enable ref counting when displaying, adds some compile time to the document
